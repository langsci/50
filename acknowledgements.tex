\chapter*{Acknowledgments}
\addcontentsline{toc}{chapter}{Acknowledgments}

In 1989 I started to study physics at the University of Groningen, because at that time it seemed to me that the working of the brain could best be explained with a physics background. Human intelligence has always fascinated me, and I wanted to understand how our brains could establish such a wonderful feature of our species. After a few years I got disappointed in the narrow specialisation of a physicist. In addition, it did not provide me the answers to the question I had. Fortunately, the student advisor of physics, Professor Hein Rood introduced to me a new study, which would start in 1993 at the University of Groningen (RuG). This study was called ``cognitive science and engineering'', which included all I was interested in. Cognitive science and engineering combined physics (in particular biophysics), artificial intelligence, psychology, linguistics, philosophy and neuroscience in an technical study in intelligence. I would like to thank Professor Rood very much for that.

This changed my life. After a few years of study, I became interested in robotics, especially the field of robotics that Luc Steels was working on at the {\scshape ai} lab of the Free University of Brussels. In my last year I had to do a research project of six months resulting in a Master's thesis. I was pleased to be able to do this at Luc Steels' {\scshape ai} lab. Together we worked on our first steps towards grounding language on mobile robots, which formed the basis of the current PhD thesis. After receiving my MSc degree (\textit{doctoraal} in Dutch) in cognitive science and engineering, Luc Steels gave me the opportunity to start my PhD research in 1997.

I would like to thank Luc Steels very much for giving me the opportunity to work in his laboratory. He gave me the chance to work in an extremely motivating research environment on the top floor of a university building with a wide view over the city of Brussels and with great research facilities. In addition, his ideas and our fruitful discussions showed me the way to go and inspired me to express my creativity.

Many thanks for their co-operation, useful discussions and many laughs to my friends and (ex-)colleagues at the {\scshape ai} lab Tony Belpaeme, Karina Bergen, Andreas Birk, Bart de Boer, Sabine Geldof, Edwin de Jong, Holger Kenn, Dominique Osier, Peter Stuer, Joris Van Looveren, Dany Vereertbrugghen, Thomas Walle and all those who have worked here for some time during my stay. I cannot forget to thank my colleagues at the Sony CSL in Paris for providing me with a lot of interesting ideas and the time spent during the inspiring off-site meetings: Fr\'ed\'eric Kaplan, Angus McIntyre, Pierre-Yves Oudeyer, Gert Westermann and Jelle Zuidema.

Students Bj\"orn Van Dooren and Michael Uyttersprot are thanked for their very helpful assistance during some of the experiments. Haoguang Zhu is thanked for translating the title of this thesis into Chinese.

The teaching staff of cognitive science and engineering have been very helpful for giving me feedback during my study and my PhD research, especially thanks to Tjeerd Andringa, Petra Hendriks, Henk Mastebroek, Ben Mulder, Niels Taatgen and Floris Takens. Furthermore, some of my former fellow students from Groningen had a great influence on my work through our many lively discussions about cognition: Erwin Drenth, Hans Jongbloed, Mick Kappenburg, Rens Kortmann and Lennart Quispel. Also many thanks to my colleagues from other universities that have provided me with many new insights along the way: Ruth Aylett, Dave Barnes, Aude Billard, Axel Cleeremans, Jim Hurford, Simon Kirby, Daniel Livingstone, Will Lowe, Tim Oates, Michael Rosenstein, Jun Tani and those many others who gave me a lot of useful feedback.

Thankfully I also have some friends who reminded me that there was more in life than work alone. For that I would like to thank Wiard, Chris and Marcella, Hilde and Gerard, Herman and Xandra and all the others who somehow brought lots of fun in my social life.

I would like to thank my parents very much for their support and attention throughout my research. Many thanks to my brother and sisters and inlaws for being there for me always. And thanks to my nieces and nephews for being a joy in my life

Finally, I would like to express my deepest gratitude to Miranda Brouwer for bringing so much more in my life than I could imagine. I thank her for the patience and trust during some hard times while I was working at a distance. I dedicate this work to you.
\\
\\
\noindent Brussels, November 2000
