\chapter{PDL Code}\label{a:pdl}

In this appendix the PDL program that runs on the physical part of the robots is presented. Its purpose is to sketch how the behaviour-based cognitive architecture presented in chapter \ref{ch:lg} can be implemented in PDL.

The language game scenario that the program implements has partly been introduced in chapter \ref{ch:lg}. The scenario is extended with the part in which the robots can find each other autonomously, as discussed in more detail in \cite{steelsvogt:1997,vogt:1997}.


The program is adapted such that there is more readability than the original program. This means that some debug facilities are left out, as well as some by-passes, which have been made to solve some not yet understood peculiarities of the robots' behaviours. In addition, although no actual implementation has been made in PDL, it is sketched how the cognitive part of the language game may be implemented in PDL. This sketch, however, leaves away the critical parts of the segmentation, discrimination and naming processes.


PDL has been introduced in chapter \ref{ch:robots}. PDL is implemented in ANSI C. Part of the code is in C, but it should be readable also for non-C programmers. Remarks are given between \texttt{/*} and \texttt{*/} or behind a double slash: \texttt{//}. PDL processes are defined as functions, which are usually preceded by a \texttt{void}.


\section*{The Code}
{\scriptsize\begin{verbatim}
/*Here are some include files with libraries for PDL, SMBII and C.
  Also some definitions and declarations are present.
  These are all left out for clarity.
*/


#define NoPulse 25
#define Relax 15

const float ON=200.0f,
  OFF=1.0f;

/*PDL declarations of quantities. 
  The network is constructed at the end of the program.
*/

quantity Identity;
quantity LFBumper, RFBumper, LBBumper, RBBumper;
         //Left- and Right- Front and Back Bumpers
quantity L0,L1,L2,L3; //The white light sensors.
quantity LeftIR, FrontIR, RightIR;
quantity LM, RM;//Left- and Right Motors
quantity IREm0,IREm1,IREm2,IREm3;//IR emitters

/*Basic behavior processes

  Touch Based Obsacle Avoidance (TBOA) */

void touch_based_obstacle_avoidance(){
  int T,DL=1,DR=1;	
  
/*If both front bumpers are active one of the directions to turn
  is randomly chosen. The appropriate direction DL(eft) or DR(ight)
  is set to 0. */

  if ((value(LFBumper))&&(value(RFBumper))){
    T=random(2);
    if (T)
      DL=0;
    else
      DR=0;
  }

/*If LFBumper and MotTBOA (motivation for obstacle avoidance
  are active, then LM:=-Retract and RM:=-LargeRetract
  For the RFBumper it is the other way around*/

  add_value(LM, 
    -DL*value(LFBumper)*MotTBOA*(Retract+value(LM)));
  add_value(RM, 
    -DL*value(LFBumper)*MotTBOA*(LargeRetract+value(RM)));

  add_value(LM, 
    -DR*value(RFBumper)*MotTBOA*(LargeRetract+value(LM)));
  add_value(RM, 
    -DR*value(RFBumper)*MotTBOA*(Retract+value(RM)));
  
/*If one of the back-bumpers is pressed and its motivation (MotTBOA) 
  is on, the motor-values are set to the default speed DS.*/

  add_value(LM, value(LBBumper)*MotTBOA*(DS-value(LM)));
  add_value(RM, value(LBBumper)*MotTBOA*(DS-value(RM)));

  add_value(LM, value(RBBumper)*MotTBOA*(DS-value(LM)));
  add_value(RM, value(RBBumper)*MotTBOA*(DS-value(RM)));

}

/*Rotation with RotateSpeed in direction of MotRot.
  If MotRot=1, then robot turns right. 
  If MotRot=-1, then robot turns left.*/

void rotate(void){
  add_value(LM,MotRot*(RotateSpeed-value(LM)));
  add_value(RM,-MotRot*(RotateSpeed+value(RM)));
}


/*Active IR Obstacle Avoidance (AOA) and IR-taxis
  Since the IR modulation for AOA is the inverse of IR-taxis, 
  the same process is used. So, if MotIRT=1, then IR-taxis is
  applied, and if MotIRT=-1, AOA is applied.

  The IR modulation is regulated in the IR emission module ...
    
  inv_sigmoid dampens the difference between the left and right IR, 
  so that large differences in IR don't give too large differences 
  in motorvalues.
*/

float inv_sigmoid(float x){
  float Alpha=2.5,Beta=0.3;
  return (1/(1+exp(Alpha*(x-Beta))));
}

void IRTaxis(void){
  float F=inv_sigmoid(abs((value(RightIR)-value(LeftIR)))/255.0);
  add_value(LM,0.1*F*MotIRT*(value(RightIR)-value(LeftIR)));
  add_value(RM,-0.1*F*MotIRT*(value(RightIR)-value(LeftIR)));
}

/*When MotFW is on, the robot will try to accelerate towards default speed.
  The motor-values increase asymptotically towards this DS.*/

void towards_default(void){ 	
  add_value(LM,MotFW*(DS-value(LM))/20.0);
  add_value(RM,MotFW*(DS-value(RM))/20.0);
}	

/*If MotStop=1, the motor-values become zero.*/

void stopMotors(void){	
  add_value(LM,-MotStop*value(LM));
  add_value(RM,-MotStop*value(RM));
}

/*Pulsing the IR. During their default behavior the robots emit pulses
  of IR so that the robots can detect each other's presence.*/

void pulse_IR(void){
  if (PulseIR){
    if ((Timer%(Pulse+NoPulse))>Pulse)
      IR=ON;
    else
      IR=OFF;
  }
}

/*Emitting the IR. The modulation of the IR is set to 1 
  if the IR=1 (this means the IR is OFF), then the robot 
  can detect other IR sources.
  If the IR > 1, the modulation is set to 95, so the robot 
  can AND emit IR AND detect its own IR.
*/

void emitIR(void){	
  if (IR>1)
    SMB2IRModDc(95);
  else
    SMB2IRModDc(1);
  add_value(IREm0,(IR-value(IREm0)));
  add_value(IREm1,(IR-value(IREm1)));
  add_value(IREm2,(IR-value(IREm2)));
  add_value(IREm3,(IR-value(IREm3)));
}


/*Sending radio messages.

  The 0 gives the reliability bit (i.e. unreliable transmission)
  receipient is the identity of the receiving robot.
  strlngth is the length of the message in bytes.
  out_buffer is the message to be send. This is specified where
  necessary.*/

void send_message(void){
  if (SEND){
    radio_link_tx(0,receipient,strlngth,(unsigned char *)out_buffer);
    SEND=0;
  }
}

/*timing increments the Timer and evaluates whether or not the
  robot is in a particular state too long. If so, the robot will
  stop its current behavior and will return to the default state.
*/

void timing(void){
  Timer++;
  if ((Timer>MaxTime)&&((StateHearer)||(StateSpeaker))){
    StateSpeaker=0;
    StateHearer=0;
    Timer=0;
  }
}

/*Reading radio linked messages */

void read_message(void){
  int i;
  Expression=0;
  while (radio_link_rx((unsigned char *)in_buffer)){
    message_length=(int)in_buffer[0];
    who_from=(int)in_buffer[1];
    for (i=1;i<=message_length;i++)
      message[i-1]=(char)in_buffer[i+1];
    Expression=1;
  }
}

/*Segmentation: In the actual implementation, the sensor-data is
  transmitted to a radio base station, connected to a PC. 
  How the segmentation is done is discussed extensively in chapter 3.
  The actual segmentation takes place off-line and the implementation 
  details are left out here for readability reasons.
*/

void segmentation(){
  if (SEGMENTATION){
    /*Read and process values of quantities L0, L1, L2 and L3.
      This process includes sensing and featue extraction.
      For each segment found, the feature values are calculated,
      the segment is added to the context and NrOfSegments is 
      incremented.
    */
  }
}

/*The processes discrimination_game, production, understanding and
  feedbackadaptation are all processed off-board. They are given 
  for completeness, however the implementation details are left out. 
  For the details, see chapter 4.
*/

void discrimination_game(){
  if (DG){
    set_of_concepts=discriminate(Topic);
    Discriminated=1;
    DG=0;
  }
}

void production(){
  if (Produce){
    Utterance=production(Topic);
    Produced=1;
    Produce=0;
  }
}

void understanding(){
  if (Decode){
    Association=decode(Utterance);
    Decoded=1;
    Decode=0;
  }
}

void feedbackadaptation(){
  if (FeedbackAdaptate){
    Feedback=interpret(Association);
    adapt_lexicon(Association);
    Adapted=1;
    FeedbackAdaptate=0;
  }
}

/*So far (for now) the cognitive processes, which are mentioned for
  completeness. The code continues with on-board processing.

  maximize() detects a maximum in the IR flow of the front IR sensor.
  Note that it is a function and not a process.*/

int maximize(){
  int N=0;
  float diff;
  diff=value(FrontIR)-PreviousIR;
  
  if ((diff<=0)&&(Previous>0)&&
      (value(FrontIR)>IRClose))
    N=1;
  Previous=diff;
  PreviousIR=value(FrontIR);
  return N;
}

/*speaker() implements the finite state automaton of the speaker.

  Each state is divided in a set op actions and final conditions.
  The actions are executed by the basic behaviors or processes as 
  defined above.

  The basic behaviors are activated by setting the motivations to 
  their appropriate values. The motivations are specified with MotXX, 
  where XX refers to the particular behavior.

  For reasons of clarity not all motivations are continously given.
  Unless otherwise specified, all motivations are initially set to 0.
  After changes, when a motivation is not given, its values is as
  last specified.

  The final conditions of one state are either the pre-conditions of
  the next state, or the pre-condition of state 0 in which the 
  default_behavior process (see below) take over. The latter type of
  final condition is modeled by MaxTime and the process timing.
  
  MaxTime specifies how long a robot may remain in a particular state.

*/

void speaker(void){
  int i,j,m,Flag;
  switch(StateSpeaker){	
  case 0:{/*Default state. See default_behavior().*/
    break;
  }
  case 1:{
    /*Waiting for confirmation and after the IR-switch has been 
      relaxed, the speaker can determine in which direction to 
      turn. Orienting towards the hearer helps finding it.*/
    //Actions.
    MotStop=1;
    IR=OFF;
    MotTBOA=1;
    
    if (Expression)
      if (strcmp(message[0],"confirm")==0)
        Confirm=1;
   
    //Final conditions
    MaxTime=RotationTime;
    if ((Timer>Relax)&&(Confirm)){
      if (value(LeftIR)>value(RightIR))
        MotRot=-1; /*Turning left*/
      else
        MotRot=1;  /*Turning right*/
      StateSpeaker=2;
      Timer=0;
    }
    break;
  }
  case 2:{/*Initial orientation*/
    //Actions
    IR=OFF;
    MotStop=0;
    MotTBOA=0; /*If robot rotates, obstacle avoidance is nasty side-effect*/

    //Final conditions
    if (Timer>(RotationTime-1)){
      /*If robot has not found other robot yet, 
        keep searching using taxis in next state.
        Since the MaxTime in this state is
        RotationTime, the transition has to be made
        before, otherwise the state will be 
        timed-out by the process `timing'*/
      StateSpeaker=2;
      Timer=0;
    }
    if (maximize()){
      if (value(FrontIR)<IRCloseEnough)
        StateSpeaker=3;/*Distance to hearer is too big.*/
      else
        StateSpeaker=4;/*Distance to hearer is ok.*/
      MotStop=1;/*Setting motor-values to 0.*/
      Timer=0;
    }
    break;
  }
  case 3:{/*Get closer to the hearer by using IR taxis.*/
    //Actions
    MotFW=1;
    MotIRT=1;
    MotStop=0;
    MotRot=0;
    IR=OFF;
    MotTBOA=1;

    //Final condition
    MaxTime=SearchTime;
    if (value(FrontIR)>IRCloseEnough){
      StateSpeaker=4;
      Timer=0;
    }
    break;
  }
  case 4:{/*Final alignment back-to-back*/
    //Actions
    MotFW=0;
    MotIRT=0;
    MotStop=0;
    MotTBOA=0;
    MotRot=1;
    IR=OFF;

    //Final conditions
    MaxTime=RotationTime;

    /*If the robot detects enough IR with its LeftBackIR sensor,
      it stops. It stands still approximately when it is facing
      the opponent robot backwards.
      Using taxis would be more sophistigated, but takes longer and
      is more error-prone.
      Since the hearer must transfer to the next state simultaneously, 
      the speaker also sends a message.*/

    if (value(LeftBackIR)>IRCloseEnough){
      StateSpeaker=5;
      strcpy(out_buffer,"aligned");
      SEND=1;
      Timer=0;
    }
    break;
  }
  case 5:{/*The hearer does its alignment. The speaker waits while 
            emitting IR.*/
    //Actions
    MotStop=1;
    MotRot=0;
    IR=ON;
    MotTBOA=1;
    
    //Final condition
    MaxTime=2*RotationTime;
    if ((Expression)&&(strcmp(message,"aligned")==0)){
      /*There is a message saying the hearer aligned successfully.
        The speaker can start the sensing.*/
      MotStop=1;
      StateSpeaker=6;
      Timer=0;
    }
    break;
  }
  case 6:{/*The speaker does its sensing, segmentation and feature
            extraction. 
            Here starts the process as described in the book.*/
    //Actions
    MotStop=0;
    MotRot=1;
    IR=0.0f;
    MotTBOA=0;

    SEGMENTATION=1;

    //Final condition
    MaxTime=2.5*RotationTime;

    /*After a particular time, the speaker stops rotating when it
      again detects the IR with its LeftBack sensor.*/

    if ((Timer>(1.75*RotationTime))&&
        (value(LeftBack)>IRCloseEnough)){
      StateSpeaker=7;
      strcpy(out_buffer,"perceived");
      SEND=1;//This way the hearer transits state as well.
      Timer=0;
    }
    break;
  }
  case 7:{/*Now the hearer does its sensing, segmentation and
            feature extraction.*/
    //Actions
    MotStop=1;
    MotRot=0;
    IR=ON;
    MotTBOA=1;

    //Final conditions
    MaxTime=3*RotationTime;

    /*Hearer finished sensing. The cognitive part can start.*/

    if ((Expression)&&(strcmp(message,"perceived")==0)){
      StateSpeaker=8;
      Timer=0;
    }
    break;
  }
  case 8:{/*The rest of the speaker's FSA is how it would look like
            if the cognitive part is processed on-board.
            For clarity the details are left out. See chapters 3 and 4 
            for details.*/
    //Actions
    MotStop=1;

    if (!Discriminated){
      Topic=random(NrOfSegments);
      DG=1;
    }
    else{//Final condition
      Discriminated=0;
      if (NrOfConcepts>0)
        StateSpeaker=9; /*Discimination game was success.*/
      else{/*Discimination game was a failure.
             Language game ends in failure.
             Ontology adapted during discrimination game.*/
        StateSpeaker=0;
        strcpy(out_buffer,"failure");
        SEND=1;
        LG++;
      }
      Timer=0;
    }
    break;
  }
  case 9:{/*The speaker's word-form production.*/
    //Actions
    MotStop=1;

    if (!Produced)
      Produce=1;
    else{//Final condition
      Produced=0;
      if (strcmp(Utterance,"nil")){
        strcpy(out_buffer,Utterance);
        /*The speaker produced an meaningful utterance*/
        StateSpeaker=10;
      }
      else{/*The speaker could not produce an utterance.
             Adaptation (word-creation) has already been 
             done during production.*/
        strcpy(out_buffer,"failure");
        StateSpeaker=0;
        LG++;
      }
      SEND=1;
    }
    break;  
  }
  case 10:{/*Feedback and Adaptation.*/
    //Actions
    MotStop=1;
    if (Expression){
      /*Hearer provided feedback, which needs to be interpreted.
        After that the lexicon can be adapted.*/
      FeedbackAdaptate=1;
    }
    
    //Final condition
    if (Adapted){
      StateSpeaker=0;
      LG++;
      Adapted=0;
    }
    break;
  }
  }
}

/*hearer() implements the finite state automaton of the hearer.*/

void hearer(void){
  int i,j,Flag;
  switch(StateHearer){
  case 0:{/*Default state. See default_behavior().*/
    break;
  }
  case 1:{/*Hearer starts waiting until speaker aligned.*/
    //Actions
    MotStop=1;
    MotTBOA=1;
    IR=ON;

    //Final conditions
    MaxTime=AlignTime;
    if ((Expression)&&(strcmp(message[0],"aligned")==0)){
      StateHearer=2;
      Timer=0;
    }
    break;
  }
  case 2:{/*The hearer has to wait for the IR to relaxate.
            Otherwise the robot cannot detect IR of the other.*/
    //Actions
    MotStop=1;
    MotTBOA=1;
    IR=OFF;

    //Final condition
    if (Timer>Relax){
      Timer=0;
      StateHearer=3;
    }
    break;
  }
  case 3:{/*Rotation for alignment.*/
    //Actions
    MotRot=1;
    MotStop=0;
    MotTBOA=0;
    IR=OFF;

    //Final conditions
    MaxTime=RotationTime;
    if (value(LeftBackIR)>IRCloseEnough){
      StateHearer=4;
      strcpy(message,"aligned");
      SEND=1;//The speaker has to transit state as well.
      Timer=0;
    }
    break;
  }
  case 4:{/*The speaker does its sensing, segmentation and
            feature extraction; the hearer waits.*/
    //Actions
    MotStop=1;
    MotRot=0;
    MotTBOA=1;
    IR=ON;

    //Final conditions
    MaxTime=3*RotationTime;
    if ((Expression)&&(strcmp(message,"perceived")==0)){
      StateHearer=5;
      Timer=0;
    }
    break;
  }
  case 5:{/*The hearer does its sensing, segmentation and
            feature extraction.*/
    //Actions
    MotStop=1;
    MotRot=1;
    IR=OFF;
    MotTBOA=0;
    SEGMENTATION=1;

    //Final conditions
    MaxTime=2.5*RotationTime;
    if ((Timer>(1.75*RotationTime))&&
        (value(LeftBack)>IRCloseEnough)){
      StateHearer=6;
      strcpy(out_buffer,"perceived");
      SEND=1;//The speaker has to transit state as well.
      Timer=0;
    }
    break;
  }
  case 6:{/*The hearer plays discrimination games for each segment
            in its context.*/
    //Actions
    MotStop=1;

    if (Topic<NrOfSegments){
      Topic++;
      DG=1;
    }
    else{//Final condition
      StateHearer=7;
      Timer=0;
    }
    break;
  }
  case 7:{/*The hearer waits for the speaker's utterance.*/
    //Actions
    MotStop=1;
    
    //Final conditions
    MaxTime=ProductionTime;
    if (Expression){
      if (strcmp(message,"failure")){
        /*The speaker produced an utterance.*/
        strcpy(Utterance,message);
        StateHearer=8;
      }
      else{/*The speaker failed either to produce or discriminate
             hence the language game fails and is finished for the hearer.*/
        StateHearer=0;
        LG++;
      }
      Timer=0;
    }
    break;
  }
  case 8:{/*The hearer tries to understand the speaker's utterance.*/
    //Actions
    MotStop=1;

    if (!Decoded)
      Decode=1;
    else{//Final condition
      Decoded=0;
      StateHearer=9;
    }
    
    break;
  }
  case 9:{/*Feedback and adaptation.*/
    //Actions
    //No physical actions are specified.

    if (!Adapted)
      FeedbackAdaptate=1;
    else{//Final condition
      Adapted=0;
      strcpy(out_buffer,Feedback);
      StateHearer=0;
      LG++;
    }
    break;
  }
  }
}

/*default_behavior describes the robots' behavior when they are 
  exploring their environment 'arbitrary' in order to find each
  other.
  When one robot finds another contact is made, and the robots 
  enter the first state in either the speaker- or hearer mode.
*/

void default_behavior(void)
{	
  if ((!StateSpeaker)&&(!StateHearer)){
    //Actions
    MotStop=0;
    MotFW=1;
    MotIRT=-1;/*Inverse IR-taxis is Active Obstacle Avoidance*/
    MotRot=0;
    Pulse=1;/*The robots pulse IR to detect each other.*/
    MotTBOA=1;

    //Final conditions
    if ((value(FrontIR)>Threshold)&&
        ((Timer%(Pulse+NoPulse)>(Pulse+Relax)))){
      /*The robot is sure it does not detect reflected IR of itself*/
      strcpy(message,"communicate");
      SEND=1;
      StateSpeaker=1;
    }
    if ((Expression)&&(strcmp(message,"communicate")==0)){
      strcpy(message,"confirm");
      SEND=1;
      StateHearer=1;
      Timer=0;
    }
  }
} 

/* Initializing the robot and some of its variables */

void initialize(void)
{
  if (Init>0){
    if (value(Identity)==1){
      //Each robot has its own identity, which is automatically detected.
      Pulse=25;
      //Some other initializations.
    }
    else{
      //Each robot has its own identity, which is automatically detected.
      Pulse=25;
      //Some other initializations.
    }	
    StateSpeaker=0;
    StateHearer=0;
    //Some other initializations.
    Init++;
  }
}

/*The main program of PDL. Here the PDL network is initialized,
  defined and constructed.
*/

void main(void){

  printf("Starting Execution\n\r");
  init_pdl();

  /*Quantities are added to the network.
    Each quantity has a name, an upper value,
    a lower value and an initial value.
  */
     
  //Sensors:
  Identity = add_quantity("Identity", 1.0,0.0,0.0);
  LFBumper = add_quantity("LeftFrontBumper", 1.0,0.0,1.0);
  RFBumper = add_quantity("RightFrontBumper", 1.0,0.0,1.0);
  LBBumper = add_quantity("LeftBackBumper", 1.0,0.0,1.0);
  RBBumper = add_quantity("RightBackBumper", 1.0,0.0,1.0);
  L0 = add_quantity("WhiteLight0",255.0,0.0,0.0);
  L1 = add_quantity("WhiteLight1",255.0,0.0,0.0);
  L2 = add_quantity("WhiteLight2",255.0,0.0,0.0);
  L3 = add_quantity("WhiteLight3",255.0,0.0,0.0);
  LeftIR = add_quantity("LeftIR",255.0,0.0,0.0);
  FrontIR = add_quantity("FrontIR",255.0,0.0,0.0);
  RightIR = add_quantity("RightIR",255.0,0.0,0.0);
  LeftBackIR = add_quantity("LeftBackIR",255.0,0.0,0.0);

  //Actuators:
  LM = add_quantity("LeftMotor",100.0,-100.0,0.0);
  RM = add_quantity("RightMotor",100.0,-100.0,0.0);  
  IREm0 = add_quantity("IREm0",200.0,0.0,0.0);
  IREm1 = add_quantity("IREm1",200.0,0.0,0.0);
  IREm2 = add_quantity("IREm2",200.0,0.0,0.0);
  IREm3 = add_quantity("IREm3",200.0,0.0,0.0);

  //Connections with the SMBII are made.

  connect_sensor(SID_BIN1, LFBumper);
  connect_sensor(SID_BIN2, RFBumper);
  connect_sensor(SID_BIN3, LBBumper);
  connect_sensor(SID_BIN7, RBBumper);
  connect_sensor(SID_BIN4, Identity);

  connect_sensor(SID_AN0,L0);
  connect_sensor(SID_AN1,L1);
  connect_sensor(SID_AN2,L2);
  connect_sensor(SID_AN3,L3);
  
  connect_sensor(SID_IR1,LeftIR);
  connect_sensor(SID_IR2,FrontIR);
  connect_sensor(SID_IR3,RightIR);
  connect_sensor(SID_IR4,LeftBackIR);
  
  connect_actuator(AID_MOTOR2,RM);
  connect_actuator(AID_MOTOR1,LM);
  
  connect_actuator(AID_IREM0,IREm0);
  connect_actuator(AID_IREM1,IREm1);
  connect_actuator(AID_IREM2,IREm2);
  connect_actuator(AID_IREM3,IREm3);

  //Processes are added to the network.

  add_process("initialize",initialize);
  add_process("touch_based_obstacle_avoidance",
	      touch_based_obstacle_avoidance);
  add_process("rotate",rotate);
  add_process("IRTaxis",IRTaxis);
  add_process("towards_default",towards_default);
  add_process("stopMotors",stopMotors);
  add_process("emitIR",emitIR);
  add_process("read_message",read_message);
  add_process("send_message",send_message);
  add_process("timing",timing);
  add_process("segmentation",segmentation);
  add_process("discrimination_game",discrimination_game);
  add_process("production",production);
  add_process("understanding",understanding);
  add_process("feedbackadaptation",feedbackadaptation);
  add_process("speaker",speaker);
  add_process("hearer",hearer);
  add_process("default_behavior",default_behavior);

  /*The PDL program is run. This program is implemented as
    an infinite loop.*/

  run_pdl(-1L);

}
\end{verbatim}}
