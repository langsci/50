\chapter{Glossary}

\begin{description}

\item[Actual success] The actual success is a measure that calculates the average success of the past 100 language games. A language game is successful when both robots of a language successfully identified a symbol that has the same form and stands for the same referent. Often this is the same as the communicative success. In these cases, the latter measure will be used.\index{actual success}

\item[Binary subspace] A binary subspace is a region in the feature space. Binary subspaces are constructed by splitting another subspace in two equal halves in one dimension of the feature space. A binary subspace is a possible definition of a category.\index{binary!subspace}

\item[Categorisation] Categorisation is the process in which a feature vector is related to one or more categories. In the experiments a category is defined by either a prototypical category or a binary subspace.\index{categorisation}

\item[Category] A category is defined by a region in the feature space.\index{category}

\item[Communicative success] The communicative success is a measure that calculates the average success of the past 100 language games as evaluated by the robots themselves. This need not be the same as the actual success.\index{communicative success}

\item[Context] A context is the set of segments identified from a single sensing event.\index{context}

\item[Consistency] Consistency is a measure that indicates how consistent the referents are named by some word-forms.\index{consistency}

\item[Discrimination] Discrimination is a process where the robot identifies categories that relate to one feature vector, but not to another feature vector from the same view. This discrimination takes place at the category level. Harnad (1990) defines discrimination directly at the level of perception. Unless mentioned otherwise, the term {\it discrimination} is used at the category level.\index{discrimination}\index{symbol grounding problem!discrimination}

\item[Discriminative success] The discriminative success is a measure that calculates the average success of the discrimination games of a robot in the past 100 language games. If a robot does not play a discrimination game in a language game, the discrimination game is considered to be a failure.\index{discriminative success}

\item[Distinctive category] A distinctive category is a category that relates to some segment in a context, but not to any other segment in the same context.\index{distinctive category}

\item[Distinctiveness] Distinctiveness is a measure that indicates to what degree the meanings used by a robot relates to the same referent as before.\index{distinctiveness}

\item[Feature] A feature is a value between $[0,1]$ that designates a property of the sensed segment.\index{feature}

\item[Feature extraction] Feature extraction calculates some property of the sensed segment from the sensor data. It reduces the complexity of the segment and returns a set of values that designate features.\index{feature!extracion}

\item[Feature space] The feature space is an n-dimensional space where each dimension is related to some property that can be calculated from the sensory data that a robot can sense. In the experiments described here there are 4 dimensions, each relating to a property of a sensory channel. The domain of each dimension are real values between $[0,1]$.\index{feature!space}

\item[Feature vector] A feature vector is an n-dimensional vector in the feature space that has as its elements the different features that are extracted from a segment. This way a feature vector is related to a segment.\index{feature!vector}

\item[Feedback] Feedback is the process where the robots evaluate the effectiveness of a language game, i.e. whether both robots communicated the same referent. The feedback evaluated can be both positive as negative.\index{feedback}

\item[Form] A form is an arbitrary string of characters from an alphabet.\index{form}

\item[Iconisation] Iconisation is the forming iconic representations. It is a term that Harnad (1990) identifies as a subpart of solving the symbol grounding problem.\index{symbol grounding problem!iconisation}

\item[Identification]  Harnad (1990) defines identification as the invariant categorisation of sensing a real world phenomenon. In this book this means that both robots successfully related the referent to a meaning and a form. Although the meaning can be different, the referent and form must be the same for both robots.\index{symbol grounding problem!identification}

\item[Joint attention] With joint attention is meant the state where both participants of a language game know the topic prior to the verbal communication.\index{joint attention}

\item[Lexicon] A lexicon is the set of form-meaning pairs that a robot has stored in its memory.\index{lexicon}

\item[Meaning] In the theory of semiotics meaning is the sense that is made of the symbol. The meaning arises in the interpretation of the symbol. In the experiments this is the category that a robot used in a language game to name a referent.\index{meaning}

\item[Parsimony] Parsimony is a measure that indicates to what degree a referent gives rise to the use of a unique meaning.\index{parsimony}

\item[Polysemy] Polysemy is the notion that a form is used to name more than one referent.\index{polysemy}

\item[Prototype] A prototype is defined as a point in the n-dimensional feature space and it is used for defining a category.\index{prototype}

\item[Prototypical category] A prototypical category is a category that is represented in the feature space by a prototype. It is defined by the region of which the points in the feature space are nearest to the prototype.\index{category!prototypical}\index{prototypical category|see{category!prototypical}}

\item[Referent] A referent is that what the symbol ``stands for". In the experiments, the referents in the robots' environment are light sources.\index{referent}

\item[Segment] A segment is a set of data from the sensory channels that is obtained by segmentation.\index{segment}

\item[Segmentation] Segmentation is the process in which the aim is, given a sensed data set, to construct regions that corresponds directly to a real world object. It is implemented by a process that identifies from a sensed data set connected areas that are uniform in some way.\index{segmentation}

\item[Sensing] Sensing is the process in which a robot records a  view of its surroundings. In the experiments described here, the robots make a full 360$^o$ turn while recording. Sensing results in a set of data points given on the sensory channels.\index{sensing}

\item[Sensory channel] A sensory channel is the channel in which the numeric output of a particular sensor flows.\index{sensory channel}

\item[Specificity] Specificity is a measure that indicates to what degree forms are used to name a unique referent.\index{specificity}

\item[Symbol] The definition of a symbol is adopted from C.S. Peirce's theory on semiotics. A symbol is defined as the relation between a referent, a meaning and a form. This relation is often illustrated with the semiotic triangle. \index{symbol}

\item[Synonymy] Synonymy is the notion that one referent is named by more than one form.\index{synonymy}

\item[Topic] The segment that is the subject of a discrimination game and/or language game is called the topic.\index{topic}

\item[Word-form] See {\it form}.

\end{description}







