\subsection{Conjunctions and single categories}

\todo{verwerk nieuwe data expstlrsel0}

It is obvious from the experimental results so far that conceptualizing over single categories does not yield an optimal ontology. A more optimal ontology for communication is reached when a concept is constructed from a conjunction of categories that are activated accross different sensory channels. Intuitively this is no surprise, since the robots' task is to conceptualize a light source's highth in contrast to other light sources' highth. When a comparing one category with another, it is useful to take all relevant features in consisderation.

The choice of taking a one sized concept set or a conjunction has been preprogrammed in the experiments reported so far. But what if the robots can choose themselves? Will they also show a preference for conjunctions of categories over single sized concept sets? The experiment presented in this section investigates whether this happens or not.

So, the categorization now not only yield single sized concept sets, but also concepts that have categories at each sensory channel. In the latter case, the concepts are constructed as in the basic experiment, so all categories of a concept are at the same hierarchical layer.

\section{Selecting categories}

\todo{Nieuw experiment -> nwe data (stlrselspA)}
In the previous experiment the speaker took a list of distinctive concepts, ordered by their meaning score. When the speaker found a matching WM association, it produced the utterance belonging to it. In this experiment, the speaker searches for all categories matching WM associations and selects the one with the highest association score for the utterance. Since all meaning scores are 0, the hearer also has no information on how good a meaning is. 

\begin{table}
\begin{tabular}{||l|c|c|c||}
\hline\hline
Score & Avg & Stdev & Std\\\hline

\hline
\end{tabular}
\caption{The results of combining single sized concept sets and conjunctions of concept sets.}
\label{t:cat:sel}
\end{table}

Table \ref{t:cat:sel} shows the results of the experiment. The communicative success and discriminative success are higher than both experiments presented before. In comparison with the basic experiment (B) the CS is better with a significance of $p=0.0078$, in comparison with the single sized concepts experiment (SSC) the significance is $p=0.0000$. In comparison to both experiment B and SSC, the discriminative success is better with a significance of $p=0.0000$. The discriminative success is higher than in the previous experiments, since there are now more configurations of categories that may be distinctive. Such increase naturally reflects on the communicative success although to a lesser extend.

A slight difference is found for the distinctiveness. It is lower than in the basic experiment with a significance of $p=0.0.0040$ and it is higher than for the SSC ($p=0.0016$). Specificity is better than both B and SSC ($p=0.0288$ and %p=0.0004$ resp.). Both parsimony and consistency show values between B and SSC with significance for parsimony $p=0.0004$ (B) and $p=0.0002$ (SSC), and for consistency $p=0.0028$ (B) and $p=0.0002$ (SSC). These measures already show what could be predicted. The communication system is better than the communication system for SSC, since it also use conjunctions of categories to conceptualize. On the other hand the communication system is worse than the basic experiment, since it also has single sized concepts that are less useful for communication.

To see if the robots actually prefer to use conjunctions of categories at the basis of their concepts, figure \ref{f:cat:comp} is illustrative.


