\chapter{Categorization}\label{ch:cat}

In this chapter experiments are presented that indicate some influences that categorization and conceptualization can have on lexicon development. The starting point will be the basic experiment that has been reported in the previous chapter. By changing some methods of the categorization process the experiments reveal some interesting results. In each experiment only one method will be varied to ensure reliable comparisons with the basic experiment. 

\p
Before the variations are introduced, a deeper investigation of the categorization process from the basic experiment will presented in section \ref{s:cat:evol}. In this experiment categories are shifted towards perceptions after a language game has been successful. Section \ref{s:cat:noshift} investigates what happens when the categories that are constructed no longer shift towards the central tendencies of the perception. 

The categories up to now did not allow any overlap at one hierarchical layer of a sensory channel. However it is plausible that categories do overlap (e.g. \cite{aitchison:1994,lakoff:1987}). Section \ref{s:cat:fs} discusses what happens when categories may overlap each other in the sensory space. 

The use of prototypical categories is different from the binary tree representations used in the original formulation of the discrimination games. Does the system yield similar results when the categories are represented by binary trees? This question will be answered in section \ref{s:cat:bin}.


\section{The evolution of categories}\label{s:cat:evol}
\index{prototype!dynamic|(}
\index{prototype!hierarchy of -|(}

In the previous chapter the basic experiment has been presented. The categorization process appears to work well, although the communicative success was pretty poor. It is interesting to see how the ontology of prototypical categories evolve in time.

\begin{figure}
\centering
\subfigure[sc0-1]{\psfig{figure=categorization//cats0-1.eps,width=5.6cm}}
\subfigure[sc1-1]{\psfig{figure=categorization//cats1-1.eps,width=5.6cm}}\\
\subfigure[sc2-1]{\psfig{figure=categorization//cats2-1.eps,width=5.6cm}}
\subfigure[sc3-1]{\psfig{figure=categorization//cats3-1.eps,width=5.6cm}}
\caption{The development of categories at hierarchical layer 1 for sensory channels sc0, sc1, sc2 and sc3.}
\label{f:cat:evol1}
\end{figure}

Figure \ref{f:cat:evol1} shows the evolution of category values at hierarchical layer 1 for each run. Recall that each layer $\lambda$ allows a maximum of $3^\lambda$ categories. These categories are constructed quite rapidly and their values keep changing. This means that they are used successfully to name a segment of which the feature value is equal to the category value. The lower and upper categories remain close to 0 and 1 resp., the middle values shift toward values somewhere in the middle between 0 and 1. The categories have the tendency to move towards what could be called the central tendency of the feature values for which the categories have been used successfully in the language games. 


\begin{figure}
\centering
\subfigure[sc0]{\psfig{figure=categorization//r0s0basis.eps,width=5.6cm}}
\subfigure[sc1]{\psfig{figure=categorization//r0s1basis.eps,width=5.6cm}}\\
\subfigure[sc2]{\psfig{figure=categorization//r0s2basis.eps,width=5.6cm}}
\subfigure[sc3]{\psfig{figure=categorization//r0s3basis.eps,width=5.6cm}}
\caption{The distribution of feature values sensed by robot r0 from the basic data set over all the situations.}
\label{f:cat:distr}
\end{figure}

Figure \ref{f:cat:distr} shows the distribution of feature values as they have been observed in the basic data set. It is clear that for all sensory channels a value near 0 and values of 1 are most frequent. For the other values in between there is no clear tendency in the distribution. Hence it is difficult to investigate whether the `middle' prototypes evolve towards a statistical central tendency. However, if one looks carefully, one sees that for sensory channel {\em sc0} between 0.3 and 0.8 the interval [0.5,0.6> has highest frequency. The prototype evolves towards 0.6, so this might correlate. Similar conclusions can be drawn for {\em sc1}, {\em sc2} and {\em sc3}. Highest distribution for {\em sc1} and {\em sc2} are in [0.3,0.4>; the category values go to 0.3. For {\em sc3} this is [0.4,0.6> with category value 0.4; here high frequencies are also observed in [0.2,0.4>.

\begin{figure}
\centering
\subfigure[sc0-2]{\psfig{figure=categorization//cats0-2.eps,width=5.6cm}}
\subfigure[sc1-2]{\psfig{figure=categorization//cats1-2.eps,width=5.6cm}}\\
\subfigure[sc2-2]{\psfig{figure=categorization//cats2-2.eps,width=5.6cm}}
\subfigure[sc3-2]{\psfig{figure=categorization//cats3-2.eps,width=5.6cm}}
\caption{The development of categories at hierarchical layer 1 for sensory channels sc0, sc1, sc2 and sc3.}
\label{f:cat:evol2}
\end{figure}

\begin{figure}
\centering
\subfigure[sc0-3]{\psfig{figure=categorization//cats0-3.eps,width=5.6cm}}
\subfigure[sc1-3]{\psfig{figure=categorization//cats1-3.eps,width=5.6cm}}\\
\subfigure[sc2-3]{\psfig{figure=categorization//cats2-3.eps,width=5.6cm}}
\subfigure[sc3-3]{\psfig{figure=categorization//cats3-3.eps,width=5.6cm}}
\caption{The development of categories at hierarchical layer 1 for sensory channels sc0, sc1, sc2 and sc3.}
\label{f:cat:evol3}
\end{figure}

\begin{figure}
\centering
\subfigure[sc0-4]{\psfig{figure=categorization//cats0-4.eps,width=5.6cm}}
\subfigure[sc1-4]{\psfig{figure=categorization//cats1-4.eps,width=5.6cm}}\\
\subfigure[sc2-4]{\psfig{figure=categorization//cats2-4.eps,width=5.6cm}}
\subfigure[sc3-4]{\psfig{figure=categorization//cats3-4.eps,width=5.6cm}}
\caption{The development of categories at hierarchical layer 1 for sensory channels sc0, sc1, sc2 and sc3.}
\label{f:cat:evol4}
\end{figure}

\begin{figure}
\centering
\subfigure[sc0-5]{\psfig{figure=categorization//cats0-5.eps,width=5.6cm}}
\subfigure[sc1-5]{\psfig{figure=categorization//cats1-5.eps,width=5.6cm}}\\
\subfigure[sc2-5]{\psfig{figure=categorization//cats2-5.eps,width=5.6cm}}
\subfigure[sc3-5]{\psfig{figure=categorization//cats3-5.eps,width=5.6cm}}
\caption{The development of categories at hierarchical layer 1 for sensory channels sc0, sc1, sc2 and sc3.}
\label{f:cat:evol5}
\end{figure}


\p
Figures \ref{f:cat:evol2} - \ref{f:cat:evol5} show the development of categories on hierarchical layers 2 to 5. There are an increasing number of categories on layers 2 to 4, layer 5 show a lower amount of categories. In contrast to the categories of layer 1, these categories hardly move in the category space. This is because categories at these higher levels are not frequently used successfully in the language games as has already been observed in the previous chapter.

The figures also show that the robot had difficulties in discriminating feature values between 0 and 0.5 rather than between 0.5 and 1. There are significantly more categories between 0 and 0.5 than between 0.5 and 1. This is not so surprising, since sensory channels that are right above or below most active sensory channels typically yield values between 0 and 0.5. A feature that corresponds to a certain referent always yield a value 1. At all layers there are categories with value 1. So, when a discrimination game fails, the chance is high that a new category will be constructed with a value between 0 and 0.5.

At layers 1 to 3 in most cases all possible places in the category space are occupied. The higher layers have not been filled completely. Although for most sensory channels the first categories are constructed at higher layers very early, the tendency is to fill the lower layers first.
\index{prototype!dynamic|)}
\index{prototype!hierarchy of|)}
\index{category!dynamic|see{prototype!dynamic}}

\section{Static categories}\label{s:cat:noshift}
\index{category!static|(}

In this section an experiment is presented that is completely the same as the basic experiment reported in the previous chapter, except that when a language game is successful the categories are {\em not} shifted towards the observation. So, once a category is introduced, its value is static and its sensitivity remains the same throughout the experiment unless another category is introduced that overlaps the category's sensitivity space. This way categorization may not be optimal, since the initial value of a category can be very close to an adjacent category although the central tendency of its potential influence lies further away. This makes it possible for another category to be sensitive for areas that are optimally supposed to be sensitive for the category in question. Table \ref{t:cat:avg-ns} shows the averaged results of 10 runs.

\begin{table}
\centering
\begin{tabular}{||l|c|c||}
\hline\hline
Score & Avg & Std\\\hline
CS & 0.349 & 0.017\\\hline
DS0 & 0.933 & 0.007\\\hline
DS1 & 0.935 & 0.006\\\hline
D0 & 0.959 & 0.000\\\hline
D1 & 0.959 & 0.000\\\hline
P0 & 0.841 & 0.005\\\hline
P1 & 0.835 & 0.006\\\hline
S0 & 0.813 & 0.022\\\hline
S1 & 0.810 & 0.015\\\hline
C0 & 0.793 & 0.007\\\hline
C1 & 0.793 & 0.007\\\hline
\hline
\end{tabular}
\caption{The average results of the experiment where the prototypical categories do not change over time. Again the table shows the average (Avg) and standard deviation of the different runs (Std). Scores are given for communicative success (CS), discriminative success (DS) for the two robots, distinctiveness (D), parsimony (P), specificity (S) and consistency (C).}
\label{t:cat:avg-ns}
\end{table}


When these results are compared with the results of the basic categories (table \ref{t:st:averages} on page \pageref{t:st:averages}), one can see that there are hardly any {\em significant} differences.

The discriminative success is about 1.5 \% higher with a significance of $p=0.0052$. The parsimony and consistency appears to be lower with a $p$-value of resp. $p=0.0432$ and $p=0.0354$. All other differences have no significance at all. So, although one might expect larger differences they have not been observed. If there is a function for shifting the categories it will be for increasing parsimony and consistency, but this has not been observed with much certainty.
\index{category!static|)}

\section{Fuzzy categories}\label{s:cat:fs}
\index{category!fuzzy|(}

In the implementations so far, a feature could only be categorized with exactly one category at a certain sensory channel at each hierarchical layer. However, human categorization might not be so clear cut (see e.g. \cite{aitchison:1994}). Especially around the boundaries of a category's sensitivity, the certainty of whether an observation belongs to a category becomes smaller. For instance, something that looks like a {\em cup} might be categorized by another person as a {\em vase} or {\em bowl} as has been shown by \cite{labov:1973}. {\em Family resemblance} \cite{wittgenstein:1958} is another example of such fuzziness.  These are examples at a higher level of categorization, but the same may hold at the lower level of categorization.\index{family resemblance}\index{Wittgenstein, Ludwig}

So, categories may overlap each other at their boundaries. An experiment has been set up where the categories may overlap. In the basic experiment a feature is categorized at layer $\lambda$ with the category of which the value is closest to the feature value. In the new experiment, a feature is categorized with those categories that are closer than a certain minimal distance, and if no such category exists, the feature is categorized with the category that is closest to the feature value. 

More formally, a prototype $c_k$ can be defined by filling in the attributes as previously done in section \ref{s:cm:proto}: $c_k = (sc_{i,k}, v_k, \lambda_k, \nu_k)$. Let $d_{i,k}=|V_i-v_k|$ be the absolute distance between value $V_i$ of feature $f_i$ and value $v_k$ of category $c_k$, then a segment $s_j=\{f_0,\ldots,f_{N-1}\}$ can now be related to a set of categories $\Lambda^{s_j}=\{c_0,\ldots,c_M\}$ (see also equation \ref{e:cm:proto}):

\begin{eqnarray}
\Lambda^{s_j}=&\{c_k=(sc_{i,k},v_k,\lambda_k,\nu_k)\; \mid (c_k \in O) \wedge \\\nonumber
&(\exists (f_i=sc_i\mbox{-}V_i \in s_j):(d_{i,k} \leq \Theta))\}
\end{eqnarray}

\n
where
\begin{eqnarray}
\Theta = \left \{\begin{array}{rl}
\Theta_\lambda & \mbox{if there exists such a category}\\
\min_{k,i} (d_{k,i}) & \mbox{otherwise.}
\end{array} \right.
\end{eqnarray}

\n
$\Theta_\lambda$ is the minimum distance of the boundaries of a certain category in layer $\lambda$, it is taken to be $\Theta_\lambda=3^{-\lambda}$. Naturally, comparisons between categories should be at the same sensory channels and hierarchical layers. The discrimination game is unaltered.

Instead of one category at each layer for a certain sensory channel, an observation may have several. This increases the chance that a referent is conceptualized more parsimoniously and thus increasing the communicative success. Whether or not this actually happens is seen in table \ref{t:cat:fs}. The experiment differs from the basic experiment only in the application of fuzzy sets.

\begin{table}
\centering
\begin{tabular}{||l|c|c||}
\hline\hline
Score & Avg & Std\\\hline
CS & 0.357 & 0.024\\\hline
DS0 & 0.875 & 0.014\\\hline
DS1 & 0.894 & 0.013\\\hline
D0 & 0.956 & 0.001\\\hline
D1 & 0.956 & 0.001\\\hline
P0 & 0.875 & 0.003\\\hline
P1 & 0.875 & 0.002\\\hline
S0 & 0.843 & 0.022\\\hline
S1 & 0.846 & 0.024\\\hline
C0 & 0.835 & 0.011\\\hline
C1 & 0.832 & 0.008\\\hline
\hline
\end{tabular}
\caption{The results of categorizing using fuzzy sets.}
\label{t:cat:fs}
\end{table}


Although there are more possible concepts, the discriminative success is lower than in the basic experiment ($p=0.0000$). The communicative success shows an insignificance difference ($p=0.1230$). However,  one run showed an exceptional small communicative success, namely 0.23. When throwing away this run (which is a valid statistical method) the average CS becomes $0.371 \pm 0.025$ which is different with a $p$-value of $p=0.0504$. So, although this looks better, it is hard to say whether the CS of this experiment is better. 

The distinctiveness is equal to the basic experiment (compare table \ref{t:st:averages} on page \pageref{t:st:averages}). The specificity seems to be higher than in the basic experiment and the consistency seems to be lower, but these differences are insignificant ($p=0.2176$ and $p=0.1230$ resp.). The parsimony however is slightly higher (0.02) with a significance of $p=0.0008$.

So, although the DS is lower than in the basic experiment, the fuzzy set approach does not appear to influence the quality of the communication system that emerges. 
\index{category!fuzzy|)}

\section{Binary Trees}\label{s:cat:bin}
\index{binary tree|(}

The implementation of the discrimination games in the current thesis differs from the work by Luc \citeN{steels:1996b} in the way categories are represented. Steels' agents construct binary trees of which the nodes are used to represent categories. In the work presented here the categories are represented by prototypes. In the experiment of this section, categories are represented by binary trees as explained in section \ref{s:cm:binary}. 

Like in the other experiments the concepts are constructed by taking conjunctions of categories at the same hierarchical layer. This has one main disadvantage in comparison to the prototype method. In the prototype method if an agent has constructed a category at a particular layer and sensory channel, the agent can categorize every segment at this layer and sensory channel. This is because a feature is categorized with the category that is {\em closest} to the feature value. In the binary tree method, this is not the case because then a feature value must be {\em within} the sensitivity of an existing category. Except for layer 0, the categories may not cover the complete sensory channel space.

\begin{table}
\centering
\begin{tabular}{||l|c|c||}
\hline\hline
Score & Avg & Std\\\hline
CS & 0.264 & 0.041\\\hline
DS0 & 0.893 & 0.005\\\hline
DS1 & 0.895 & 0.003\\\hline
D0 & 0.922 & 0.005\\\hline
D1 & 0.933 & 0.003\\\hline
P0 & 0.856 & 0.004\\\hline
P1 & 0.858 & 0.007\\\hline
S0 & 0.841 & 0.045\\\hline
S1 & 0.852 & 0.040\\\hline
C0 & 0.791 & 0.016\\\hline
C1 & 0.789 & 0.023\\\hline
\hline
\end{tabular}
\caption{The results of the experiment investigating the binary tree method.}
\label{t:cat:bin}
\end{table}

Table \ref{t:cat:bin} shows the results of 10 runs of 5,000 language games of the experiment with the binary tree method. A first thing that strikes is the lower discriminative success ($p=0.0004$) as predicted. This lower DS is because increases slower (figure \ref{f:cat:bin} (a)). It finally increases to the same level as in the basic experiment (see figure \ref{f:st:plot} at page \pageref{f:st:plot}).

\begin{figure}
\centering
\subfigure[CS]{\psfig{figure=categorization//csbin.eps,width=5.6cm}}
\subfigure[DS]{\psfig{figure=categorization//dsbin.eps,width=5.6cm}}\\
\subfigure[S]{\psfig{figure=categorization//sbin.eps,width=5.6cm}}
\subfigure[D]{\psfig{figure=categorization//dbin.eps,width=5.6cm}}\\
\subfigure[C]{\psfig{figure=categorization//cbin.eps,width=5.6cm}}
\subfigure[P]{\psfig{figure=categorization//pbin.eps,width=5.6cm}}
\caption{The evolution of the measures for the binary tree method.}
\label{f:cat:bin}
\end{figure}

The communicative success stays well behind the CS of the basic experiment ($p=0.0016$). It is not directly clear why the CS is about 8.5 \% lower. The CS appears to stop learning after 1,000 language games. The consistency is about 0.025 lower than in the basic experiment ($p=0.0354$), but parsimony is $\pm 0.005$ higher ($p=0.0524$). Because the $p$-values are relatively high, it is difficult to assign meaning to these differences.

Most obvious difference is the distinctiveness, which is $\pm 0.03$ lower with a significance of $p=0.0002$. In contrast to previous plots that have been shown the distinctiveness does not grow towards a value of 1, but it stabilizes around 0.97 (figure \ref{f:cat:bin} (d)). So, it seems that the concepts less reliably refer to the corresponding referents, thus indicating more conceptual polysemy. Specificity is higher indicating that referential polysemy is not worse. This observation, however, has a low significance: $p=0.1904$. Hence lower distinctiveness might explain a lower communicative success, since a word-meaning no longer refers to one referent, which it does when $D=1$.

\p
That communicative success of the binary tree method stays well behind the basic experiment is probably due to a combination of lower discriminative success and distinctiveness. But this has not been confirmed with the specificity and consistency. Additional causes could lie in the lower parsimony and consistency, however it is too speculative to deduct such conclusions from the measured figures, since their $p$-values are relative high.
\index{binary tree|)}

\section{Summary}

In this chapter the influence of different categorization strategies have been investigated. These different strategies investigated for instance what happened when categories are static. Another experiment investigated the influence of applying the idea of fuzzy sets to the categorization process. The binary tree method has been compared with the prototype method. Additionally, a deeper inquiry has been made in the development of categories in the basic experiment as presented in the preceding chapter. Figure \ref{f:cat:results} gives an overview of the results compared to the basic experiment.

\begin{figure}
\subfigure[CS]{\psfig{figure=categorization//rescs.eps,width=5.6cm}}
\subfigure[DS]{\psfig{figure=categorization//resds.eps,width=5.6cm}}\\
\subfigure[S]{\psfig{figure=categorization//resspec.eps,width=5.6cm}}
\subfigure[D]{\psfig{figure=categorization//resdist.eps,width=5.6cm}}\\
\subfigure[C]{\psfig{figure=categorization//rescons.eps,width=5.6cm}}
\subfigure[P]{\psfig{figure=categorization//respars.eps,width=5.6cm}}
\caption{The average scores of the experiments discussed in this chapter: static categories (NS), fuzzy sets (FS) and binary trees (BIN) compared with the basic experiment (B).}
\label{f:cat:results}
\end{figure}

\p
In section \ref{s:cat:evol} the evolution of dynamic categories has been shown as they were constructed in the basic experiment. The evolution showed how the categories emerged in time and how they shift their values towards a central tendency of the observations that the robots could name successfully. When the categories do not shift towards this central tendency, the communication system that is developed is qualitatively about the same. So, in the current set up, the dynamics of categories does not add any other functionality than possibly a more realistic model.

There is psychological and linguistic evidence that categories overlap making way for a fuzzy edge phenomenon. An experiment has been done where categories can overlap near their edges (or even near their center when they are very close to each other). The discriminative success is significantly worse than in the basic experiment and the communicative success improves slightly but this is not very significant. It was predicted that parsimony and consistency would improve since the robots get a better chance to choose concepts or word-forms more consistently. This however, has only been observed for the parsimony; the difference in consistency was insignificant.

The binary tree method is performing worse than the basic setup, which is a bit surprising. The communicative success is lower than chance and the discriminative success is lower than originally. The latter observation has much to do with a slow start. Remember that a feature cannot always be categorized if its value does not fit inside a valid interval, whereas in the prototype a feature value can always be categorized with the closest category. Even if this category belongs to the other side of the sensory space. The system's distinctiveness is significantly different, but for the other measures parsimony, specificity and consistency the differences are much less significant or maybe not at all.

\p
Now that the conceptualization of the system is understood better, it is interesting to see how different parameters and methods influence the lexicon formation. The next chapter investigates the influence of various physical conditions on the ontological and lexical development.
